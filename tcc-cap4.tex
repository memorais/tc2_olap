\chapter{Materiais utilizados}
	
\section{Banco de simulações de docagem}

\section{Receptor Investigado}

\section{Ligantes Considerados}

\section{Script para preparação dos dados}

Com objetivo de automatizar o processo de preparação dos dados de simulações de docagem molecular, foi criado um script utilizando a linguagem Python. Este script é responsável pela sumarização dos dados resultante da simulação para posteriormente alimentar o modelo OLAP com os dados já organizados.

Através de uma fórmula matemática, faz-se um filtro para obtenção das ligações mais estáveis, enquanto as ligações que apresentam valores ruins, consideradas instáveis, são descartadas.

Os dados de entrada são carregados através de dois arquivos delimitados por vírgula (comma-separated values), onde cada um deles deve conter os resultados da simulação de docagem com as coordenadas do LIGANTE (?). Após a execução do script, tem-se como resultado os dados sumarizados e organizados, prontos para alimentar o modelo OLAP.

O procedimento executado por este script consiste nas seguintes etapas:

\begin{enumerate}
    \item bla
    \item bla
    \item bla
\end{enumerate}

mostrar cálculo  que é realizado para obter apenas os melhores resultados.
mostrar tabela de exemplo do resultado de saída do script.