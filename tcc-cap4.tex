\chapter{Materiais utilizados}
	
\section{Banco de simulações de docagem}

A execução da simulação de docagem é realizada por vários algoritmos, e para este caso tem como resultado um arquivo CSV que contém cada estado, também chamado de snapshot, da execução. Cada linha deste arquivo se refere a um snapshot e contém todos os resíduos, todos os ligantes envolvidos e qual foi o melhor FEB e RMSD dos ligantes para aquele determinado snapshot. Este banco de simulações possui 3100 snapshots que resultam em 3101 linhas no arquivo, levando em consideração o cabeçalho, e 12335 colunas.

\section{Receptores Investigados}

\section{Ligantes Considerados}

\section{Scripts para preparação dos dados}

Com objetivo de automatizar o processo de preparação dos dados de simulações de docagem molecular, foram criados dois scripts utilizando as linguagens Python e Bash. O primeiro script é responsável pela sumarização dos dados resultante da simulação para posteriormente alimentar o modelo OLAP com os dados já organizados. O segundo script foi criado para manipular as colunas de um arquivo CSV contendo o resultado dos processos de docagem.

Através da distância euclidiana ($d(P, Q)= \xsqrt{(x - a)^{2} +(y - b)^{2} + (z - c)^{2}}$), faz-se um filtro para obtenção das ligações mais estáveis, enquanto as ligações que apresentam valores ruins, consideradas instáveis, são descartadas.

Os dados de entrada no primeiro script de carga recebem três parâmetros. O primeiro parâmetro é um arquivo texto contendo a lista de resíduos, o segundo é um arquivo texto contendo a lista de ligantes e o terceiro é o arquivo delimitado por vírgula (CSV) contendo o resultado da simulação dos processos de docagem molecular. Após a execução deste script, tem-se como resultado os dados sumarizados e organizados, prontos para alimentar o modelo OLAP. O segundo script é responsável por excluir ou manter determinadas colunas dos arquivos CSV. Ele recebe como primeiro parâmetro uma lista de colunas separadas por vírgula, o segundo parâmetro prove as opções para excluir (-x) ou manter apenas (-m) as linhas e o terceiro é o arquivo CSV a ser manipulado.

O procedimento de execução destas rotinas pode ser definido na seguinte sequência:

\begin{enumerate}
    \item Remoção de espaços em branco do arquivo contendo o processo de simulação da docagem. 
    \item Criação das listas de resíduos e ligantes levando em consideração o arquivo com a simulação.
    \item Execução do script de manipulação do arquivo de simulação de docagem para separar os resíduos NAH e o ligante TCL, conforme determinação do especialista de domínio.
    \item Execução do script para avaliação dos melhores resultados.
    \item Sumarização dos resultados.
\end{enumerate}

mostrar tabela de exemplo do resultado de saída do script.
