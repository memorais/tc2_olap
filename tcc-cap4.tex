\chapter{Materiais utilizados}

\section{Receptor Investigado}

A Tuberculose, causada pela bactéria \emph{Mycobacterium tuberculosis}, é a segunda doença infecciosa que mais causa mortes em todo o mundo. O tratamento com os principais fármacos utilizados atualmente tem duração, em média, de 6 meses e apresenta diversos efeitos colaterais. Muitas vezes o paciente interrompe precocemente o tratamento, propiciando o surgimento de bactérias resistentes aos compostos químicos. Em função disso, O desenvolvimento de novos fármacos ou a melhoria de compostos químicos já existentes é um esforço necessário para controle e combate a Tuberculose.

Dentro deste cenário, o LABIO da PUCRS realiza pesquisas e estudos \emph{in-silico} utilizando como receptor a enzima InhA da \emph{Mycobacterium tuberculosis} e através de experimentos de docagem molecular procura identificar potenciais candidatos à fármaco. A enzima em estudo é essencial para a sobrevivência da bactéria \emph{Mycobacterium tuberculosis} e se torna alvo para realização de experimentos\cite{kar11}.

Em sua estrutura, a proteína InhA contém um conjunto de 268 resíduos, que por sua vez são compostos por um total de 4.008 átomos.

\section{Ligantes Considerados}

% -----------------------
%
% TO DO: descrever ligantes TCL e ETH
%
% -----------------------

\section{Banco de simulações de docagem}

Primeiramente, os especialistas de domínio do LABIO executaram dois experimentos de docagem molecular, cada um deles envolvendo um ligante diferente. Para o primeiro experimento, foi considerado o TCL como ligante. Já para o segundo, foi utilizado o ligante ETH. Ambos experimentos empregaram a enzima InhA como receptora, e sua flexibilidade foi representada por um conjunto de 3.100 \emph{snapshots} resultantes de simulações por DM. 

Cada ligante foi submetido a 3.100 simulações de docagem, uma para cada conformação do conjunto. Os atributos calculados por estes experimentos foram FEB e RMSD, e após a execução, os dados resultantes dos experimentos foram exportados para um \emph{dataset} em formato CSV (\emph{Comma-separated values}).

Neste \emph{dataset} os dados estão organizados da seguinte maneira: cada linha representa um dos 3.100 \emph{snapshots} da proteína receptora; nas colunas estão os valores das diferentes posições tridimensionais de cada um dos átomos da proteína, e o posicionamento final de cada ligante com seus respectivos valores de FEB e RMSD.

No \emph{dataset} utilizado, cada átomo está representado por três colunas, cada uma delas contendo respectivamente as coordenadas X, Y e Z de seu posicionamento. A Figura \ ilustra o conteúdo dos dados existentes no \emph{dataset} utilizado neste trabalho.

% -----------------------
%
% TO DO: colocar figura ou montar tabela para ilustrar as informações contidas no dataset.
%
% -----------------------

\section{Scripts para preparação dos dados}

Com objetivo de automatizar o processo de preparação dos dados de simulações de docagem molecular, foram criados dois scripts utilizando as linguagens Python e Bash. O primeiro script é responsável pela sumarização dos dados resultante da simulação para posteriormente alimentar o modelo OLAP com os dados já organizados. O segundo script foi criado para manipular as colunas de um arquivo CSV contendo o resultado dos processos de docagem.

Através da distância euclidiana ($d(P, Q)= \sqrt{(x - a)^{2} +(y - b)^{2} + (z - c)^{2}}$), faz-se um filtro para obtenção das ligações mais estáveis, enquanto as ligações que apresentam valores ruins, consideradas instáveis, são descartadas.

Os dados de entrada no primeiro script de carga recebem três parâmetros. O primeiro parâmetro é um arquivo texto contendo a lista de resíduos, o segundo é um arquivo texto contendo a lista de ligantes e o terceiro é o arquivo delimitado por vírgula (CSV) contendo o resultado da simulação dos processos de docagem molecular. Após a execução deste script, tem-se como resultado os dados sumarizados e organizados, prontos para alimentar o modelo OLAP. O segundo script é responsável por excluir ou manter determinadas colunas dos arquivos CSV. Ele recebe como primeiro parâmetro uma lista de colunas separadas por vírgula, o segundo parâmetro prove as opções para excluir (-x) ou manter apenas (-m) as linhas e o terceiro é o arquivo CSV a ser manipulado.

O procedimento de execução destas rotinas pode ser definido na seguinte sequência:

\begin{enumerate}
    \item Remoção de espaços em branco do arquivo contendo o processo de simulação da docagem. 
    \item Criação das listas de resíduos e ligantes levando em consideração o arquivo com a simulação.
    \item Execução do script de manipulação do arquivo de simulação de docagem para separar os resíduos NAH e o ligante TCL, conforme determinação do especialista de domínio.
    \item Execução do script para avaliação dos melhores resultados.
    \item Sumarização dos resultados.
\end{enumerate}

mostrar tabela de exemplo do resultado de saída do script.
