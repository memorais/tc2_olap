\chapter{Desenvolvimento do modelo OLAP}
%-------------------------------------------------------
% descrever como foi feito o levantamento dos residuos mais interesantes. 
%  Explicar que a lista dos mais presentes foi comparada com uma lista originalmente desenvolvida por um especialista de dominio. 
%  Encontrou se casos que estavam e nao estavam na lista, os que nao apareciam na lista constituem de residuos que se ficam ˜proximos”ao sitio de ligação, mas que na realidade estão fora do sítio.
% descrever como  adicionar novas dimensoes baseadas em novos residuos
%-------------------------------------------------------

Para possibilitar a construção do modelo OLAP, primeiramente foi necessário identificar e definir as questões de negócio que seriam relevantes sob o ponto de vista do especialista de domínio. Em virtude disso, o levantamento destes dados foi feito por meio de entrevistas com os especialistas do LABIO.

A Tabela \ref{tab:questaoNegocio} descreve as questões de negócio que foram identificadas como sendo mais relevantes para os experimentos de docagem realizados no LABIO. 

\begin{table}[h]
\caption{Estrutura do \emph{dataset} contendo os dados resultantes de experimentos de docagem molecular para os ligantes TCL e ETH.}
\label{tab:questaoNegocio}
\centering
\begin{tabular}{@{}ll@{}}
\toprule
\textbf{\#} & \multicolumn{1}{c}{\textbf{Definição de questões de negócio que a modelagem OLAP deve ajudar a solucionar}} \\ \midrule
\textbf{1.} & Associar um grupo para cada conformação;                                                                    \\
\textbf{2.} & Identificar o comportamento das conformações baseado nas métricas de FEB e RMSD;                            \\
\textbf{3.} & Identificar conformações/grupos que apresentam o maior número de contatos com os ligantes;                  \\
\textbf{4.} & Com base no item 3, identificar quais são os resíduos mais importantes;                                     \\
\textbf{5.} & Ainda com base no item 3, identificar quais grupos apresentaram melhores valores de FEB e RMSD.             \\ \bottomrule
\end{tabular}
\end{table}

\section{Identificação de métricas}

Durante as entrevistas realizadas, pode-se perceber que informações baseadas nas métricas de FEB e RMSD possuíam grande relevância para responder as questões de negócio. Entretanto foi necessário definir certas propriedades e limitações de valores para adequar os cálculos às necessidades do negócio. Dessa maneira, todas as definições citadas neste capítulo foram estabelecidas em conjunto com os especialistas de domínio do LABIO para que os resultados apresentados pudessem representar a realidade.

O cálculo da FEB é um dos métodos utilizados pelos softwares de docagem que permite avaliar a interação receptor-ligante. Quanto menor for o resultado deste cálculo, mais favorável é a ligação estabelecida. 

Na maioria dos experimentos, os melhores resultados de FEB são valores negativos. Portanto, qualquer conformação que apresente valores de FEB positivos não foram levados em consideração. Dessa maneira evita-se que os resultados positivos venham a interferir em uma análise futura dos valores agregados de um experimento. 

O cálculo do RMSD é utilizado para obter a distância média entre os átomos. Nos experimentos de docagem, este cálculo é feito para comparar o posicionamento inicial do ligante com o posicionamento final após a execução da docagem.

Tanto a FEB quando o RMSD dão uma visão para o especialista de quão satisfatório foi o processo de docagem para uma determinada iteração. Enquanto a FEB mede a qualidade da docagem no aspecto termodinâmico da questão, o RMSD tem como natureza avaliar geometricamente como estão dispostas as moléculas do resíduo e do ligante.

Além disso, para ser possível responder o item 3 da Tabela \ref{tab:questaoNegocio}, houve a necessidade de inclusão de uma métrica para contabilizar o número de contatos entre o ligante e um resíduo.

%-------------------
% TO DO: explicar como foi feito o cálculo para obter o número de contatos
%        descrever limitação de 2 e 4 angstrons.
%-------------------


\section{Identificação dos resíduos relevantes}
Um dos pontos mais importantes para responder aos questionamentos dos especialistas de domínio e fundamental para composição das dimensões, era saber quais eram os resíduos mais relevantes. A enzima InhA possui 268 resíduos \cite{KARANADUNOSM09} e o processo de identificação dos mais importantes leva em consideração o número de contatos do resíduo com o ligante. Na base de informações que foi resultado do processo de simulação de docagem, cada resíduo é representado por uma tripla contendo a sua localização espacial nos eixos x, y e z.

De acordo com o especialista de domínio, para ser considerado contato do resíduo com o ligante é preciso estar em uma distância entre 2 e 4 angstrom. Valores inferiores a 2 angstrom são descartados por serem considerados uma sobreposição. Desta forma é utilizado o cálculo da distância euclidiana entre os átomos do resíduo e do ligante \cite{KARANADUNOSM09} para encontrar estes valores, desprezando os átomos de hidrogênio conforme determinação dos especialistas de domínio. Quanto maior o número de contatos do resíduo com o ligante, mais relevante para o especialista é o resíduo. As regras para elencar os resíduos mais relevantes podem ser sumarizadas da seguinte forma:

\begin{enumerate}
    \item Distância de 2 a 4 angstrom são considerados contatos.
    \item Distância inferior a 2 angstrom são desconsideradas. 
    \item Descarte do cálculo para os átomos de hidrogênio.
    \item Quanto mais contatos, mais relevante.
\end{enumerate}

Após execução dos algoritmos elaborados dentro do trabalho para coletar estas informações, foram identificados 15 resíduos mais relevantes que posteriormente foram avaliados pelos especialistas e reduzidos a uma lista de 10 resíduos. Com base em uma lista anteriormente elaborada por um especialista de domínio, conforme mostrado na Figura \ref{fig:ListaOsmar}, os 5 resíduos que sobraram dos 15 iniciais foram identificados como casos onde a proteína, devido a flexibilidade, abriu uma cavidade acima da cavidade do substrato, permitindo ligações do ligante fora da cavidade do substrato.

\begin{figure}[h]
        \center
        \includegraphics[width=12cm]{images/ListaProfOsmar.png}
        \label{fig:ListaOsmar}
        \caption{Lista dos principais resíduos do Professor Osmar}
\end{figure}

A Figura \ref{fig:PlotResiduos} mostra os 15 resíduos que foram identificados inicialmente. Aqueles indicados em vermelho são os considerados importantes e os indicados em branco foram descartados porque ficaram fora do sítio de ligação.

\begin{figure}[h]
        \center
        \includegraphics[width=10cm]{images/avaliacao_Residuos_nomes.png}
        \label{fig:PlotResiduos}
        \caption{Lista inicial dos 15 resíduos mais relevantes}
\end{figure}

Com base nestas prerrogativas foram selecionados os 10 principais resíduos mais relevantes para responder às questões dos especialistas conforme mostrado.

\begin{itemize}
	\item PHE\_148 (Phenylalanine)
	\item ILE\_193 (Isoleucine)
	\item GLY\_95 (Glycine)
	\item THR\_195 (Threonine)
	\item ILE\_94 (Isoleucine)
	\item MET\_198 (Methionine)
	\item MET\_160 (Methionine)
	\item SER\_19 (Serine)
	\item ILE\_15 (Isoleucine)
	\item TYR\_157 (Tyrosine)
\end{itemize}


\section{Dimensões}
	mostrar hierarquias de dimensoes
	
\section{Construção do modelo no Analisys Services}
	exibir a tabela pivotante com os valores medios (feb,rmsd)
