\chapter{Conclusão}

O uso de um modelo OLAP para análise de dados de experimentos de simulações de docagem molecular é uma abordagem nova. Apesar do conjunto de dados utilizados para o desenvolvimento deste trabalho ter sido relativamente pequeno, contendo apenas 3.100 conformações para cada um dos dois ligantes, o modelo OLAP construído se mostrou eficiente para analisar os dados através de uma perspectiva multidimensional.

Certamente com o uso contínuo desta solução, por um maior período de tempo e utilizando uma exaustiva massa de informações de experimentos de docagem, o modelo OLAP possibilitará ao especialista de domínio a obtenção de informações relevantes, podendo identificar padrões de comportamento baseado no histórico dos experimentos e outras análises relevantes que seriam inviáveis de serem executadas de forma manual.

Os processos de extração, transformação e carga dos dados desenvolvidos utilizando a linguagem Python foram bem sucedidos e mostraram eficiência durante as execuções. Apesar de estarem separados em arquivos distindos, cada um deles visando atender à uma necessidade específica de negócio, podem ser reutilizados para outros experimentos de docagem sem que seja necessário grande alterações.

Como contribuição para a comunidade de bioinformática e com o LABIO da PUCRS, este projeto desenvolvido ficará disponível para os especialistas de domínio utilizarem nos experimentos realizados.

A abordagem da utilização de um modelo OLAP para análise de experimentos de docagem molecular não parece ser usual, entretanto, com a sua eficácia comprovada, a tendência é que isto possa ser utilizado para cruzar dados de forma bastante flexível, situação que hoje ainda limita os especialistas.

