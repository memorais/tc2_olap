\chapter{Conclusão}

A modelagem OLAP para analisar experimentos de simulação de docagem molecular se mostrou eficiente conforme visto no capítulo \ref{cap:ResultadosObtidos}. Embora o conjunto de dados utilizado tenha sido relativamente pequeno, com 3100 conformações para cada um dos dois ligantes, o modelo OLAP construído deve funcionar conforme esperado para \emph{data sets} ainda maiores. Os processos de extração, transformação e carga dos dados utilizando a linguagem de programação Python foram bem sucedidos conforme visto no capítulo \ref{cap:DesenvolvimentoDoModeloOLAP}, e podem ser reutilizados em outros projetos de docagem sem necessidade de grandes alterações. 

A abordagem de utilização do modelo OLAP para análise de experimentos de docagem molecular não parece ser usual entretanto, com a sua eficácia comprovada, a tendência é que isto possa ser utilizado mais massivamente para cruzar os dados da simulação de forma bastante flexível, situação que hoje ainda limita os especialistas.
