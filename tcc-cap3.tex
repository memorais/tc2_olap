% ----------------------------------------------------------
% Ferramentas de Desenvolvimento
% ----------------------------------------------------------
\chapter{Ferramentas de Desenvolvimento}

\section{Linguagem Python}
A escolha da linguagem Python foi uma peça-chave para possibilitar o desenvolvimento de um script para preparação dos dados utilizados neste trabalho. Uma simulação de docagem molecular gera um alto volume de dados, e para tratá-los, foi necessário a utilização de uma linguagem que suportasse esta massa de dados e apresentasse uma forma simples de desenvolvimento.

O Python é uma linguagem na qual possui uma curva de aprendizado relativamente simples se comparada com as outras linguagens utilizadas no mercado, isto se deve ao fato de o Python possuir uma estrutura de dados mais alto nível e uma abordagem simples, mas efetiva para programação orientada à objeto. Por possuir uma sintaxe clara e de tipagem dinâmica, se torna simples a compreensão \cite{pyt00}. 

Outra característica importante desta linguagem é ser multiplataforma. Dessa forma, é possível a execução em ambientes Windows, Mac e todas as distribuições Linux. Esta característica possibilita que programas escritos nesta linguagem sejam portáveis para qualquer outra plataforma com facilidade \cite{pyt01}.

A biblioteca padrão da linguagem Python possui módulos nativos para processamento de texto e expressões regulares. Observando à estas características, o Python apresentou-se como uma linguagem ideal para ser empregada neste trabalho.

\section{Microsoft SQL Server Analysis Services}
O Microsoft SQL Server Analysis Services é um banco de dados multi dimensional OLAP, que também pode ser utilizado para data mining, e trabalha em cima do banco de dados relacional Microsoft SQL Server \cite{SIVSTE05}. Esta ferramenta foi utilizada para o desenvolvimento do cubo, bem como para toda criação de estrutura de tabelas e população dos dados.

Assim como outras ferramentas OLAP, o Analysis Services é utilizado dentro das organizações principalmente para auxiliar no processo de tomada de decisão. Como o modelo OLAP permite um método de acessar e analisar dados com grande flexibilidade, as empresas viram nestas características uma plataforma perfeita para integração e consolidação de informações gerenciais.

A ferramenta é de implementação rápida, já que todo o processo de instalação stand-alone da solução não leva mais que 50 minutos, o Analysis Services assim como a linguagem de programação Python, possui uma curva de aprendizado relativamente simples. Outro aspecto que contribuiu para adoção desta tecnologia foi a parceria Microsoft DreamSpark com a PUCRS que permite a utilização desta ferramenta sem a necessidade de compra das suas licenças de uso.

explicar didaticamente sobre o que é e pra que serve
