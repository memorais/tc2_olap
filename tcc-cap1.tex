\chapter{Introdução}

olap vai possibilitar a analise atraves de diversas perspectivas
possibilitar ao especialista de domínio uma agilidade na identificação das melhores ligações

\section{Caracterização do Problema}
O processo de desenvolvimento de novos fármacos é uma atividade fundamental para a indústria farmacêutica, na qual é utilizado não somente para o desenvolvimento de novos compostos para atender uma determinada necessidade, mas também tem como objetivo evoluir os compostos já existentes em busca de um resultado aprimorado e mais eficaz. 

Este processo exige o envolvimento de profissionais e organizações das mais diversas áreas, como companhias de biotecnologia, autoridades reguladoras, pesquisadores acadêmicos, entre outros. Além de ser uma tarefa interdisciplinar ela é consideravelmente complexa dada a natureza do comportamento existente para as ligações químicas.


Este processo é complexo e exige, em média, 14 anos desde a identificação até a liberação pelo órgão regulador. As etapas envolvidas neste processo vão desde a identificação de um possivel candidato à fármaco, seguido por uma pesquisa de otimização, testes \emph{in-silico} e \emph{in-vitro}, análises toxicológicas até os ensaios clínicos. Estima-se que os custos de desenvolvimento de um novo fármaco seja aproximadamente de 1,2 bilhões de dólares \cite{kun92}. 


\section{Motivação}

\section{Objetivos}
\subsection{Objetivo Geral}
O objetivo deste trabalho é contribuir com a análise dos resultados de simulações de docagem molecular através do desenvolvimento de um modelo OLAP, de forma a organizar estes resultados sob uma estrutura de cubos, e possibilitar, ao especialista de domínio, a realização de uma análise multidimensional das informações ali presentes e a obtenção de informações relevante de uma maneira mais rápida e menos custosa.

\subsection{Objetivos Específicos}
\begin{enumerate}
	\item Desenvolver uma ferramenta para preparação dos dados de simulações de docagem molecular.
	\item Desenvolver um modelo OLAP baseado em um banco de dados utilizado para armazenamento de resultados de docagem molecular.
	\item Permitir aos especialistas de domínio o cruzamento das informações de uma simulação por agrupamento de conformações.
	\item Contribuir para que os resultados das simulações de docagem molecular possam ser analisados multidimensionalmente, facilitando a identificação de complexos receptor-ligantes estáveis.
\end{enumerate}

\section{Organização deste Documento}